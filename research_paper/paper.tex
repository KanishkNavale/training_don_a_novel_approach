\documentclass{article}

% if you need to pass options to natbib, use, e.g.:
%     \PassOptionsToPackage{numbers, compress}{natbib}
% before loading neurips_2020

% ready for submission
% \usepackage{neurips_2020}

% to compile a preprint version, e.g., for submission to arXiv, add add the
% [preprint] option:
%     \usepackage[preprint]{neurips_2020}

% to compile a camera-ready version, add the [final] option, e.g.:
% \usepackage[final]{neurips_2020}

% to avoid loading the natbib package, add option nonatbib:
\usepackage[nonatbib]{neurips_2023}
\usepackage[utf8]{inputenc} % allow utf-8 input
\usepackage[T1]{fontenc}    % use 8-bit T1 fonts
\usepackage{hyperref}       % hyperlinks
\usepackage{url}            % simple URL typesetting
\usepackage{booktabs}       % professional-quality tabless
\usepackage{amsfonts}       % blackboard math symbols
\usepackage{amsmath}
\usepackage{nicefrac}       % compact symbols for 1/2, etc.
\usepackage{microtype}      % microtypography
\usepackage{cleveref}       % smart cross-referencing
\usepackage{graphicx}
\usepackage[english]{babel}
\usepackage{csquotes}




% Bibliography
\usepackage[%
  backend=biber,%
  backref=false,%
  giveninits=true,%
  autocite=inline,%
  sorting=none,% in order of occurence. Other option: nyt (name year title)
  sortcites=true,%
  mincitenames=1,%
  maxcitenames=2,%
  maxbibnames=10,%
  doi=true,%
  isbn=false,%
  url=false,%
  natbib=false,
]{biblatex}

\renewcommand*{\bibfont}{\small}
\addbibresource{neurips_2020.bib}

\title{Training Dense Object Nets: A Novel Approach}

\author{%
  Kanishk Navale\textsuperscript{1}\thanks{for correspondence: kanishk.navale@sereact.ai}, \quad
  Ralf Gulde\textsuperscript{1, 2}, \quad
  Marc Tuscher\textsuperscript{1}, \quad
  Oliver Riedel \textsuperscript{2}\\
  \textsuperscript{1}Sereact GmbH, Stuttgart, Germany \quad
  \textsuperscript{2}ISW, Universität Stuttgart, Stuttgart, Germany\\
}

\begin{document}



\maketitle

\begin{abstract}
  Our work proposes a novel framework that addresses the computational resource limitations associated with training Dense Object Nets (DON)
  while achieving robust and dense visual object descriptors. DON's descriptors are known for their robustness to
  viewpoint and configuration changes, but training them requires computationally expensive image pairs with correspondence mapping.
  This limitation hampers dimensionality and robustness, thereby restricting object generalization.
  To overcome this, we introduce a synthetic augmentation data generation procedure and a novel deep learning architecture
  that produces denser visual descriptors with reduced computational demands. Notably, our framework eliminates the need for
  image-pair correspondence mapping and showcases its application in a robot-grasping pipeline.
  Experimental results demonstrate that our approach yields descriptors as robust as those generated by DON.
\end{abstract}

\section{Introduction}
The objectives of long-standing robotics and robotic manipulation are to create a general-purpose robot capable of carrying out practical activities like Chappie~\cite{blomkamp2015chappie} or C-3PO~\cite{lucas1977star}. While advancements have been made recently in adjacent domains, achieving this goal remains a work in progress. For instance, AlphaGo~\cite{silver2018general},
a game-playing artificial intelligence system trained entirely on self-play, defeated the world's best human Go player at the time. Subsequently, \citeauthor{silver2016mastering}~\cite{silver2016mastering}
developed artificial intelligence algorithms that mastered the game of chess, Go, World of Warcraft~\cite{entertainment2013world}, and Shogi,
surpassing human playing expertise. Most of these algorithms learn directly from visual data, such as gameplay recordings or online video streams, emphasizing the importance of visual data in AI. Meanwhile, the launch of AlexNet~\cite{krizhevsky2017imagenet} in 2012 transformed the field of computer vision. Other visual tasks, such as semantic segmentation~\cite{long2015fully}, object identification and recognition~\cite{he2017mask}, and human pose estimation~\cite{guler2018densepose}, have also witnessed significant gains in recent years. Significant breakthroughs have been made in robotics, ranging from self-driving cars to humanoid robots capable of performing complex tasks using cameras and other vision sensors. Despite these advancements, the most frequently used robotic manipulation systems have evolved slightly in the previous 30 years. Typical auto-factory robots continue to perform repetitive operations such as welding and painting, following a pre-programmed course with no feedback from the surroundings. If we want to increase the utility of our robots, we must move away from highly controlled settings and robots that perform repetitive actions with little feedback or adaptability capabilities. Liberating ourselves from these constraints of controlled settings-based manufacturing would allow us to enter new markets, as witnessed by the proliferation of firms~\cite{sereact} competing in the logistics domain.

The ideal object representation for robot grasping and manipulation tasks remains to be engineered today.
Existing representations may not be suitable for complex tasks due to limited capabilities of understanding an object's geometrical and structural information.
In \citeyear{florence2018dense}, \citeauthor{florence2018dense}~\cite{florence2018dense} introduced a novel visual
object representation to the robotics community,  terming it ``dense visual object descriptors''. DON, an artificial intelligence
framework proposed by
\Citeauthor{florence2018dense}~\cite{florence2018dense} produces dense visual object descriptors. In detail, the DON converts every pixel in the
image ($I[u, v] \in \mathbb{R}^3$) to a higher dimensional embedding ($I_D[u, v] \in \mathbb{R}^D$) such that $D \in \mathbb{N}^+$ consuming
image-pair correspondences as input yielding pixelwise embeddings
which are nothing but dense local descriptors.
The dense visual object descriptor generalizes an object up to a certain extent and has been recently
applied to rope manipulation \cite{rope-manipulation},
block manipulation \cite{block-manipulation}, robot control \cite{florence2019self}, fabric manipulation \cite{fabric-manipulation} and
robot grasp pose estimation \parencites{kupcsik2021supervised}{adrian2022efficient}. \citeauthor{adrian2022efficient}~\cite{adrian2022efficient}
further demonstrated that DON can be trained on synthetic data and still generalize to real-world objects. Furthermore, \citeauthor{adrian2022efficient}~\cite{adrian2022efficient} demonstrated that
the quality of descriptors produced by the DON framework depends on the higher or longer embedding dimension. We tried training the DON on a computation
device equipped with NVIDIA RTX A6000 GPU with 48GB VRAM. However, we could not train the DON to produce a higher embedding dimension due to the limited VRAM.
The DON framework is computationally expensive, as shown in Table~\ref{table:don_gpu_bechmark}, and limits the user to generalize objects to a certain extent making it
difficult to use as a robot grasping pipeline in real-world logistics and warehouse automation scenarios.

\begin{table}[htb]
    \caption{Benchmark of DON framework trained on GPU with 48GB VRAM with 128 image-pair correspondences, batch size of 1 and ``Pixelwise NTXENT Loss''~\cite{adrian2022efficient} as a loss function.}
    \label{table:don_gpu_bechmark}
    \centering
    \begin{tabular}{lllll}
        \toprule
        \multicolumn{5}{c}{GPU VRAM consumption to train DON}   \\
        \midrule
        Descriptor Dimension & 3     & 8      & 16     & 32     \\
        VRAM Usage (GB)      & 9.377 & 13.717 & 20.479 & 30.067 \\
        \bottomrule
    \end{tabular}
\end{table}

To overcome the computation resource limitation to produce denser visual object descriptors, we propose a novel framework to
train and extract dense visual object descriptors produced by DON, which is computationally efficient.

\section{Related Work}
\Citeauthor{florence2018dense}~\cite{florence2018dense} introduced the Pixelwise Contrastive loss function to train DON,
which involves sampling pixels in an image-pair and computing the Contrastive loss between the pixels in the first image
and those in the second image. This optimization procedure aims to improve the descriptor based on a similarity metric.
However, the Pixelwise Contrastive loss function is computationally expensive and requires numerous matching and non-matching
image-pair correspondences to work optimally. When optimizing DON using a large number ($N$) of image-pair correspondences,
the computational resources consumed by the optimization procedure increase significantly due to the exponential growth of
pixelwise descriptor similarity comparisons ($2^N$).

In their work, \citeauthor{florence2020dense}~\cite{florence2020dense} discovered that the Pixelwise Contrastive loss
function used to train DON might yield poor performance if a computed correspondence is spatially inconsistent.
They also highlighted that the precision of contrastive-trained models could be sensitive to the relative weighting
between positive and negative sampled pixels. To address these limitations, \citeauthor{florence2020dense}
proposed a new continuous sampling-based loss function called the Pixelwise Distribution loss. This novel
loss function leverages smooth and continuous pixel space sampling instead of the discrete pixel space
sampling method employed by the Pixelwise Contrastive loss. The Pixelwise Distribution loss eliminates
the need for non-matching correspondences, leading to significant savings in computation resources.

On a different note, \citeauthor{kupcsik2021supervised}~\cite{kupcsik2021supervised} utilized Laplacian Eigenmaps~\cite{belkin2003laplacian}
to embed a 3D object model into an optimally generated embedding space, serving as the target for training DON in a supervised fashion.
However, this methodology does not reduce the computational resource consumption required to train DON. In contrast,
\citeauthor{hadjivelichkov2021fully}~\cite{hadjivelichkov2021fully} employed offline unsupervised clustering based on confidence
in object similarities to generate hard and soft correspondence labels. These labels were then used as matching and non-matching
correspondences to train DON effectively.

Building upon the concept of SIMCLR-inspired frameworks~\parencites{chen2020simple}{zbontar2021barlow},
\citeauthor{adrian2022efficient}~\cite{adrian2022efficient} introduced a similar architecture and another
novel loss function called the Pixelwise NTXent Loss. This loss function robustly trains DON by leveraging
synthetic correspondences computed from image augmentations and non-matching image correspondences.
Notably, \citeauthor{adrian2022efficient}'s experiments demonstrated that the novel loss function is
invariant to batch size variations, unlike the Pixelwise Contrastive Loss. Furthermore, it is worth
noting that most of the discussed optimization methodologies heavily rely on correspondences to train DON effectively.

Moving on to the aspect of image-pair correspondences and dataset engineering,
the DON training strategy proposed in \cite{florence2018dense, florence2020dense} relies on depth information to
compute correspondences between image pairs using camera intrinsics and pose information \cite{hartley2003multiple}.
However, when utilizing consumer-grade depth cameras to capture depth information, the resulting depth data can be noisy,
particularly when dealing with tiny, reflecting objects common in industrial environments. Noisy depth information hampers
the computation of consistent spatial correspondences in an image pair. To overcome this challenge,
\citeauthor{kupcsik2021supervised}~\cite{kupcsik2021supervised} associated 3D models of objects with image views,
effectively training DON without relying on depth information. Their approach proved efficient for smaller,
texture-less, and reflective objects. Additionally, \citeauthor{kupcsik2021supervised}
compared different training strategies for producing 6D grasps on industrial objects and
demonstrated that a unique supervised training approach enhances pick-and-place resilience in industry-relevant tasks.

In contrast, \citeauthor{nerf-Supervision}\cite{nerf-Supervision} employed NeRF\cite{mildenhall2021nerf},
a method that reconstructs a 3D scene from a sequence of images captured by a smartphone camera.
They extracted correspondences from the synthetically reconstructed scene to train DON. Remarkably,
\citeauthor{adrian2022efficient}'s experiments indicated that DON trained on synthetic data generalizes
well to real-world objects. Furthermore, they adopted the $PCK@k$ metric, commonly used in
\parencites{chai2019multi}{fathy2018hierarchical}, to evaluate and benchmark DON's performance
in cluttered scenes that were previously not extensively studied.

In our work, we do not use any loss functions as proposed in \parencites{florence2018dense}{florence2020dense}{kupcsik2021supervised}{adrian2022efficient}{hadjivelichkov2021fully}{nerf-Supervision}
to train DON. However, we adopt the network architecture from DON~\cite{florence2018dense}
as our architecture's backbone and train on the task of the KeypointNet\parencites{suwajanakorn2018discovery}{zhao2020learning}
with few network modifications. Moreover, we evaluate the descriptor's robustness produced by our framework on the $PCK@k$ metric as in comparision
to benchmarks in \cite{adrian2022efficient} as it is the only benchmark available for DON. Furthermore, we compare the computational resource consumption.

\section{Methodology}
\subsection{Dataset Engineering}

We have chosen the cap object for creating synthetic dataset as the cap mesh models are readily available in the ``Shapenet'' library~\cite{chang2015shapenet}
as it containes rich object information including textures. Furthermore, we choose 5 cap models from the Shapenet library and use
Blenderproc~\cite{blenderproc} to generate the synthetic dataset.
For each cap model we save one RGB image, mask and depth in the synthetic scene. Addtionally, we employ synthetic augmentations as proposed in \cite{adrian2022efficient}
to synthetically spatial augment cap's position and rotation in an image including background randomization
using Torchvision~\cite{marcel2010torchvision} library. To generate camera poses for different viewpoints,
an augmented image-pair is sampled randomly and image-pair correspondences is computed
\footnote[1]{GitHub Link: \\ \url{https://github.com/KanishkNavale/Mapping-Synthetic-Correspondences-in-an-Image-Pair}}as illustrated in the Figure~\ref{fig:image_augs}.
Using depth information we project the computed correspondences to camera frame and compute relative transformation between two camera-frame coordinates of the correspondences
using Kabsch's transformation~\cite{kabsch}.

\begin{figure}[htb]
    \centering
    \includegraphics[scale=0.2]{images/debug_correspondences.png}
    \caption{Depiction of image synthetic spatial augmentation and correspondences mapping in an image-pair. The colored encoded dots in the figure represents correspondences in an image-pair.}
    \label{fig:image_augs}
\end{figure}

\subsection{Framework \& Mining Strategy}

As a backbone, we employ ResNet-34 architecture \cite{resnet}.
We preserve the last convolution layer and remove the pooling and linear layers.
The backbone downsamples the RGB image $I_{RGB} \in \mathbb{R}^{H \times W \times 3}$
to dense features $I_d \in \mathbb{R}^{h \times w \times D}$
such that $ h \ll H, w \ll W \text{ and } D \in \mathbb{N}^+$.
We upsample the dense features from the identity layer
(being identity to the last convolution layer in the backbone) as illustrated in the Figure~\ref{fig:modified_dnn} in page~\pageref{fig:modified_dnn} as follows:
\begin{equation}
    f_U: I \in \mathbb{R}^{h \times w \times D} \rightarrow I_D \in \mathbb{R}^{H \times W \times D}.
\end{equation}
The upsampled dense features is extracted and treated as dense visual local descriptors produced from the DON in otherwords
we extract or mine the representations from the backbone.
Similarly as in \cite{suwajanakorn2018discovery}, we stack spatial-probability regressing layer and
depth regressing layer on top of the identity layer to predict $N \in \mathbb{N}^+$ number of keypoint's spatial-probability as follows:
\begin{equation}
    f_S: I_d \in \mathbb{R}^{h \times w \times D} \rightarrow I_s^N \in \mathbb{R}^{h \times w \times N},
\end{equation}
and depth as follows:
\begin{equation}
    f_D: I_d \in \mathbb{R}^{h \times w \times D} \rightarrow I_{\hat{d}} \in \mathbb{R}^{h \times w \times N}.
\end{equation}

We incorporate continuous sampling method $f_E$ from \parencites{florence2020dense}{suwajanakorn2018discovery}
to convert the upsampled predicted spatial-probability and depth of a keypoint to spatial-depth expectation as follows:
\begin{equation}
    f_E \circ g_E:[I_s, I_{\hat{d}}] \rightarrow [u, v, d]^T \in \mathbb{R}^3 \text{ , where }  g_E: I \in \mathbb{R}^{h \times w \times N} \rightarrow I \in \mathbb{R}^{H \times W \times N}.
\end{equation}
Furthermore, we train the framework in a twin architecture fashion as proposed in
\parencites{chen2020simple}{zbontar2021barlow}{florence2018dense}{florence2020dense}{kupcsik2021supervised}{adrian2022efficient}{hadjivelichkov2021fully}{nerf-Supervision}
on the KeypointNet task.

\begin{figure}[htb]
    \centering
    \includegraphics[scale=0.35]{images/arch.png}
    \caption{Illustration of novel framework designed to efficiently compute and seamlessly extract dense visual object descriptors.
        During inference we extract dense visual object descriptors directly from the network and ignore predicted spatial-depth expectation of the keypoints.}
    \label{fig:modified_dnn}
\end{figure}


\subsection{Loss Functions}

For training, we directly adopt silhoutte consistency loss ($\mathcal{L}_{obj}$), variance loss ($\mathcal{L}_{var}$) and separation loss ($\mathcal{L}_{sep}$) functions from \cite{suwajanakorn2018discovery} to train the network on the keypoint prediction task.
However, we modify the multi-view consistent loss and relative pose estimation loss. In the case of multi-view consistency loss we
project the predicted spatial-depth expectation using camera intrinsics as follows:
\begin{equation}
    X_{cam} \in \mathbb{R}^{3 \times 1} = \mathcal{I}_{cam}^{-1}  \ [u, v, 1.0]^T \otimes d \text{ , where  } \ \mathcal{I}_{cam} \in \mathbb{R}^{3 \times 3} \text{ and }  u, v, d \in \mathbb{R}^+.
\end{equation}

Furthermore, we project the camera coordinates of the keypoints from one camera viewpoint to another camera viewpoint using relative transformation
supplied from the synthetic augmentation procedure as follows:

\begin{equation}
    \label{eqn:mvc}
    \mathcal{L}_{mvc} \in \mathbb{R} = \mathcal{H}(\hat{X}^B_{cam}, \mathcal{T}_{A \rightarrow B} \hat{X}^A_{cam}) \text{ , where  } \hat{X}_{cam}=[X_{cam}, 1.0]^T \in \mathbb{R}^{4 \times 1} ,
\end{equation}


In Equation~\ref{eqn:mvc}, $ \mathcal{T}_{A \rightarrow B} \in SE(3) \in \mathbb{R}^{4 \times 4}$ is a Special Euclidean Group~\cite{thurston2014three} which
is relative transformation from camera-frame $A$ to camera-frame $B$. We use Huber loss $\mathcal{H}$ as it produces smoother gradients for framework optimization.
Furthermore, we do not discard the relative transformation information to calculate the realative pose loss as suggested in \cite{suwajanakorn2018discovery}
and being influenced from \cite{zhao2020learning} we modified the relative pose loss as follows:
\begin{equation}
    \mathcal{L}_{pose} = \Vert log(\mathcal{T}_{truth}^{\dagger} \mathcal{T}_{pred}) \Vert \text{ , where  } \ log: SE(3) \rightarrow \mathfrak{se}(3) \text{ and } \mathcal{T}^{\dagger} = \begin{bmatrix}
        R^T & -R^T t \\
        0^T & 1
    \end{bmatrix} \in SE(3).
\end{equation}


\subsection{Robot Grasping Pipeline}
To use the proposed framework as a robot grasping pipeline, we extract dense visual object descriptors from the network and store
one single descriptor of objects in a database manually for now. During inference, we extract dense visual object descriptors from the network and
query the descriptor from the database to find the closest match as follows:
\begin{equation}
    \label{eqn:gaussian_kernel}
    \mathbb{E}{[u^*, v^*]_{d}} = \operatorname*{argmin}_{u, v} \ exp-\left(\dfrac{\|I_D[u, v] - d\|}{exp(t)}\right)^2
\end{equation}
Where $t \in \mathbb{R}$ controls the kernel width influencing the search space to compute the optimal spatial expectation $\mathbb{E}{[u^*, v^*]_{d}}$ of
the query descriptor $d \in \mathbb{R}^D$ in the descriptor image $I_D \in \mathbb{R}^{H \times W \times D}$. The computed spatial expectation is projected to robot frame using camera intrinsics and pose to perform a pinch grasp.
Furthermore, Franka Emika 7-DOF robot manipulator with 2 jaw gripper and wrist mounted Intel Realsense D435 camera is used as testing setup as illustrated in Figure~\ref{fig:robot_setup}.

\begin{figure}[htb]
    \centering
    \includegraphics[scale=0.3]{images/franka.png}
    \caption{Illustration of the robot grasping pipeline setup. In the image, the robot is highlighted in red, the caps are highlighted in green and the camera is highlighted in blue.}
    \label{fig:robot_setup}
\end{figure}






\section{Experiments \& Results}
\subsection{Dense Object Nets}
We implemented out training and benchmarking using ``PyTorch-Lightning''\cite{falcon2019pytorch} and ``PyTorch''\cite{paszke2019pytorch} libraries.
Futhermore, we employ
ADAM\cite{kingma2014adam} optimizer to optimize the model for 2500 epochs with learning rate of
$\alpha = 3 \times 10^{-4}, \beta_1 = 0.9 \text{ and } \beta_2 = 0.999$ with weight decay $\eta =10^{-4}$ to benchmark the DON with Pixelwise NT-Xent loss as in ~\cite{adrian2022efficient}
with a fixed batch size of 1 and 128 image-pair correspondences.
As per the benchmarking results in Table~\ref{table:don_training_results}, the robustness of the descriptor increases as the dimension of the descriptor gets longer.

\begin{table}[htb]
    \caption{Benchmark of DON framework for GPU consumption and AUC for $PCK@k,  \forall k \in [1, 100]$ metric.}
    \label{table:don_training_results}
    \centering
    \begin{tabular}{lllll}
        \toprule
        \multicolumn{5}{c}{DON benchmark}                                                                     \\
        \midrule
        Descriptor Size ($D$) & $3 $              & $8 $              & $4 $              & $32$              \\
        AUC for $PCK@k$       & $0.922 \pm 0.006$ & $0.933 \pm 0.011$ & $0.948 \pm 0.012$ & $0.953 \pm 0.008$ \\
        VRAM Usage (GB)       & $9.377 $          & $13.717 $         & $20.479 $         & $30.067$          \\
        \bottomrule
    \end{tabular}
\end{table}

The AUC for $PCK@k, \forall k \in [1, 100]$ is computed with 256 image-pair correspondences and
the mectric's mean and std. deviation is calculated from benchmarking 3 DON models trained for a single descriptor dimension.
We could not train the descriptor dimension of 64 and 128 due to the limited VRAM. Furthermore, to inspect the
results of trained DON, a interface is built using the PyGame library~\cite{pygame} to visualize the results of the trained DON.
The mouse pointer in image space is mapped to the pixel and the descriptor at that pixel is queried in an another image-descriptor space.
We further use the spatial probability of the descriptor to visualize the queried descriptor
in the image space using Equation~\ref{eqn:gaussian_kernel}
to identity if there are any multi-modal spatial activations in the descriptor spaces and there are none as shown in Figure~\ref{fig:check_don}.

\begin{figure}[htb]
    \centering
    \includegraphics[scale=0.25]{images/test_don.png}
    \caption{Depiction of the spatial probability heatmaps of the descriptor in the image space. We set the temperature in the Equation~\ref{eqn:gaussian_kernel} to $1.1$
        and render the spatial probability heatmaps in the interface. The first and second image from the left and the right highlights the semantically equivalent descriptors in the image space.}
    \label{fig:check_don}
\end{figure}


\subsection{Our Framework}

To train our framework we employ ADAM optimizer to optimize the model for 2500 epochs with learning rate of
$\alpha = 1 \times 10^{-3}, \beta_1 = 0.9 \text{ and } \beta_2 = 0.999$ with no weight decay. We further use a fixed batch size of 1
and use StepLR scheduler with step size of 2500 and gamma of 0.9 to train the model. At first, we trained our model with 16 keypoints with
margin of 10 pixels as a hyperparameter for the separation loss and later we trained the models with 128 keypoints with margin of 2 pixels.


\begin{table}[htb]
    \caption{Benchmark of our framework for GPU consumption and AUC for $PCK@k,  \forall k \in [1, 100]$ metric.}
    \label{table:framework_training_results}
    \centering
    \begin{tabular}{lllll}
        \toprule
        \multicolumn{5}{c}{Our framework with 16 keypoints}                                                   \\
        \midrule
        Descriptor Size ($D$) & $64 $             & $128 $            & $256 $            & $512$             \\
        AUC for $PCK@k$       & $0.922 \pm 0.006$ & $0.933 \pm 0.011$ & $0.948 \pm 0.012$ & $0.953 \pm 0.008$ \\
        VRAM Usage (GB)       & $3.799 $          & $4.191 $          & $5.241 $          & $7.341$           \\ \hline
        \multicolumn{5}{c}{Our framework with 128 keypoints}                                                  \\
        \midrule
        Descriptor Size ($D$) & $64 $             & $128 $            & $256 $            & $512$             \\
        AUC for $PCK@k$       & $0.922 \pm 0.006$ & $0.933 \pm 0.011$ & $0.948 \pm 0.012$ & $0.953 \pm 0.008$ \\
        VRAM Usage (GB)       & $4.913 $          & $5.409 $          & $6.551$           & $7.915$           \\ \hline
        \multicolumn{5}{c}{Our framework with 128 keypoints}                                                  \\
        \midrule
        Descriptor Size ($D$) & $3 $              & $16 $             & $32 $             & $64$              \\
        AUC for $PCK@k$       & $0.922 \pm 0.006$ & $0.933 \pm 0.011$ & $0.948 \pm 0.012$ & $0.953 \pm 0.008$ \\
        VRAM Usage (GB)       & $4.913 $          & $5.409 $          & $6.551$           & $7.915$           \\
        \bottomrule
    \end{tabular}
\end{table}


\subsection{Robot Grasping Pipeline}

For the robot grasping pipeline, we trained our framework with actual caps. As the synthetic data generation only needs mask and depth information, we could
create mask in no time. Additionally, while training the framework we do not need the actual real world depth information as the framework computes it's own.
We later extracted the dense visual local descriptors from the framework and visually inspected for any insconsistencies in the descriptor space as shown in Figure~\ref{fig:check_real_caps}.
and found it to be consistent. Furthermore, we did not use the models trained on the synthetic dataset as the representations were inconsistent with the real caps.

\begin{figure}[htb]
    \centering
    \includegraphics[scale=0.15]{images/test_real_caps.png}
    \caption{Visual inspection of the dense visual descriptors space of the real caps.}
    \label{fig:check_real_caps}
\end{figure}

For robot grasping, a descriptor is picked from the descriptor space and queried in the real-time such that robot can pinch grasp the object.
We could successfully grasp the caps with the robot as shown in Figure~\ref{fig:straight_grasp}.

\begin{figure}[htb]
    \centering
    \includegraphics[scale=0.15]{images/straight_grasps.png}
    \caption{Depiction of the straight robot grasping pipeline.}
    \label{fig:straight_grasp}
\end{figure}

As our framework inertly regresses keypoints on the object, we could use it as an alternative approach to grasp the caps by computing the pose
generated by the keypoints considering the actual depth information instead of network regressed depth information. We extract the spatial probability of each keypoint
from the framework and deactivate spatial probablities where the depth information is missing as the depth image from the camera is noisy.
Furthermore, the spatial expectations of the keypoints is projected to camera frame to calculate a 6D pose in the camera frame.
The 6D pose is transformed the robot frame to perform a aligned grasp as shown in Figure~\ref{fig:aligned_grasp}.

\begin{figure}[htb]
    \centering
    \includegraphics[scale=0.17]{images/aligned.png}
    \caption{Illustration of the aligned robot grasping pipeline.}
    \label{fig:aligned_grasp}
\end{figure}

We did not evaluate the robot grasping pipeline.

\section{Conclusion}
We present a novel framework for mining dense visual object descriptors without explicitly training DON.
By leveraging synthetic augmentation data generation and a novel deep learning architecture,
our approach produces robust and denser visual local descriptors while consuming up to 86.67\% lesser computational resources than the originally proposed framework.
Furthermore, it eliminates the additional task of computing a large number of image-pair correspondences.
We demonstrate the application of our framework as a robot-grasping pipeline in two methodologies,
one of which our framework demonstrates its capabilities to produce object-specific 6D poses for robot grasping.


\section{Future Work}
The framework will be extended to produce multi-object dense visual descriptors in cluttered scenes.
Furthermore, we will start incorporating ROI layers in the framework and adding additional object classification loss functions.



\printbibliography

\end{document}
